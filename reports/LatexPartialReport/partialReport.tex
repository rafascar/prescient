\documentclass[12pt]{partialreport}

\projectTitle{Prescient: Context-Aware Home \\Monitoring and Control Central\\}
\idJems{155008}
\professorName{Antônio Augusto Medeiros Fröhlich}
\dateReport{04/06/2016}
\studentsNames{Rodrigo Schmitt Meurer\\Rafael Eriberto Mariot Scarduelli \\ Daniel Manzoni Seerig}
\university{Federal University of Santa Catarina}
\proposedSchedule {%
04/04 & Send proposal \\ 
20/05 & Receive board \\
20/05 & Start documentation \\
20/05 - 10/06 & Development of Galileo's case with board sensors \\
20/05 - 25/05 & Board Sensors Implementation and Tests \\
25/05 - 30/05 & Integration of Galileo with LISHA's Smart Room \\
01/06 - 10/06 & Integration Validation \\
10/06 - 15/07 & Hybrid application Development for smart devices control \\
15/07 - 20/07 & Implementation and Evaluation of user on network discovery \\
20/07 - 30/07 & Interface Galileo Gateway with Web and Database \\
01/08 & Start development of context-aware algorithm \\
21/08 & Start development of data visualization in hybrid application \\
20/09	& Finish data visualization in hybrid application \\
20/09	& Send partial report \\
05/10	& Conclude context-aware algorithm \\
06/10 - 16/10  & Final Tests, bug corrections and implementation validation \\ 
20/10	& Prepare for presentation \\ 
20/10 & Finish documentation  \\ 
30/10	& Submit final report \\
01/11 - 04/11 & Presentation at SBESC \\
}

\updatedSchedule{%
04/04 & Send proposal & 100 \% \\ 
20/05 & Receive board & 100 \%\\
20/05 & Start documentation & 25 \%\\
22/05 - 30/05 & Development of hybrid application base graphical interfaces based on HTML, JavaScript and Jquery Mobile & 100 \%\\
01/06 - 05/06 & Integration and validation of the graphical interface with charting libraries for data visualization & 100 \%\\
05/06 - 07/06 & Study of board capabilities and embedded operating system choice & 100 \%\\
07/06 - 10/06 & Development of smartphone detection daemon on PC& 100 \%\\
08/06 & Technical Webinar & 100 \%\\
10/06 - 20/06 & Ubilinux installation and setup on Galileo & 100 \%\\
20/06 & Setup subnet and wireless router to test the daemon on Galileo & 100 \%\\
21/06 & Primary tests for the smartphone detection in Galileo (Proof of Concept) & 100 \%\\
22/06 - 30/06 & First tests with Galileo and EPOSMoteIII (attempting to control LISHA's Smart Room lights) & 100 \%\\
01/07 - 10/07 & Evaluation of desired sensors to attach to Galileo and EPOSMoteIII & 100 \%\\
10/07 - 12/07 & Instalation and tests with wiringx86 python module for Galileo Gen2 & 100 \%\\
13/07 - 25/07 & Sensor Tests on Galileo Board with wiringx86 & 80 \%\\
30/07 & Start development of context-aware algorithm & 30 \%\\
01/08 - 08/08 & Communicate hybrid application with python in Galileo through Flask API &  80\% \\
08/08 - 12/08 & Performance evaluation for the user detection daemon & 100 \% \\
24/08 & Technical Webinar & 100 \%\\
01/09 - 10/09 & Construction of Galileo and EPOSMoteIII Case & 90 \%\\
11/09 & Galileo and EPOSMoteIII installation on Case & 100 \% \\
12/09 & Design printed circuit to handle the sensors and other peripherals & 20 \%\\
15/09 & Start construction of Smart Room Model for the Demo at SBESC & 15 \%\\
20/09 & Partial report submission & 100 \%\\
20/09 & Reinstall AC control with smart device & -\\
21/09 - 28/09 & Finish hybrid application development & -\\
29/09 - 07/10 & Train and test algorithm with real data & -\\
08/10 - 10/10 & Finish refining machine learning algorithm & - \\
11/10 & Validate results and overall implementation & - \\
11/10 & Record video for presentation & - \\
12/10 - 19/10 & Finish demo model of the smart room & - \\
20/10 & Prepare for presentation & - \\ 
20/10 - 28/10 & Finish documentation & - \\ 
30/10 & Submit final report & - \\
}

\begin{document}
\maketitle
\content

\section*{Proposed Scheduled}
\oldScheduleTable

\section*{Update Scheduled}
\newScheduleTable

\section*{Difficulties and Workarounds}
\subsection*{Situation 1}
Portability is a very important issue in Internet of Things based projects. A user must be able to have access to what he wants in the closest platform. If a user is surfing the internet on his computer it is easier to open another tab in the browser than to find the smartphone and open an app to control a device or check the current consumption of an appliance. Therefore, cross-platform implementations of the system is very important. The implementation of such, however, is not trivial. Android applications are implemented in Java while web apps are in HTML, JavaScript and CSS.

\subsection*{Solution  1}
To deal with this portability issue we chose the Apache Cordova, an open-source mobile development framework, that is able to generate cross-platform applications from HTML5, CSS3, and JavaScript. This way it is possible to extend an application across more than one platform, without having to re-implement it with each platform's language and tool set.

\subsection*{Situation 2}
Our project complies a great range of  functionalities, from controlling GPIO pins, scanning the network to search for a certain MAC address and interfacing applications through the internet. This requires many different packages and functionalities from Linux. With the yocto many times it was needed to  create a specific image with the package we wanted to install. In addition, to our application, it is much harder to develop using the arduino interface to make system calls to run linux scripts. It is much easier to develop the whole system in a single environment.

\subsection*{Solution 2}
Instead of using the linux image available on many of the galileo gen. 2 tutorials, we chose to use an alternative embedded linux distribution based on debian, called ubilinux. ubilinux is an embedded Linux distribution from Emutex, based on Debian Jessie. It is targeted at embedded devices that have limited memory and storage capabilities. This Linux distribution uses the same package manager as many other Debian based distributions, the apt-get. Therefore this distribution has access to a wide range of packages available for linux. Among them, the Nmap tool, that is used to detect a smartphone's MAC address in the network.
In addition, there is a python module developed by Emutex that provides a simple and unified API to talk to the GPIO pins in the Intel Arduino capable boards, such as Galileo Gen. 2. Up to now this module supports, among many others, writing to and reading from both digital and analog inputs/outputs from the Galileo GPIO, which is enough to satisfy all the desired functionalities of our project.

\subsection*{Situation 3}
The smart room has a range of devices that are controllable through a IEEE 802.15.4 network. These devices are connected to EPOSMoteIII, wireless, real-time microcontrollers. Since our galileo board does not have any native/integrated wireless capability, we extended it with another EPOSMoteIII. This way the Galileo board can control all the appliances forwarding commands and receiving data through the EPOSMoteIII. To interface the mote with the board we chose to use the Python Programming Language. Python has a module that encapsulates the access for the serial port making it easy to communicate the mote with the galileo board through serial communication. The choice of python as the default programming language, however, made necessary a framework to enable the communication between the python scripts and the hybrid application. This framework should also be lightweight to run in the Galileo without causing any major overheads and be able to respond to requests made by the hybrid application.

\subsection*{Solution 3}
Flask is a "microframework" primarily aimed at small applications with simpler requirements. It provides support for quickly building REST APIs. With this we were able to develop routines in Python that handle incoming requests from the hybrid application to the Galileo Board. As for the hybrid application side, it issues JQuery Ajax REST requests, enabling communication between both.

\subsection*{Situation 4}
On our first try to detect a certain smartphone's MAC address connected to the network we implemented a BASH script on a PC that used the Nmap tool to verify if it was connected or not to the network. In the PC we did not encounter any performance related problems. However, a PC processor is many times faster than an embedded such as the one used on the Galileo Board. Since user detection is a process that must be running with high frequency to detect if the user is present or not in the network, when running the script on the Galileo it made the processor busy for a long time, reducing the overall performance of the system. This performance issue is specially a problem when the user is present in the room, because the system needs to keep constantly adjusting the devices to the user preferences. On the other hand, when there are no users connected to the network, the system can use the Nmap tool to discover a possible reconnection without need to worry with the processing time overhead.

\subsection*{Solution 4}
In order to deal with this performance issue we noticed that instead of always checking the entire network for the given MAC address we could ping the respective IP address once the Galileo discovers the IP to which the device is connected. In our tests we found out that by using the ping command we were able to considerably reduce the processing overhead, reducing the time needed to check if a user is still connected to the network or disconnected from it. If the user disconnects from the network we can resume using Nmap to search for any other user connection.


\subsection*{Situation 5}
The choice of the machine learning algorithm used in this project is considered a milestone because it affects directly the performance, prediction accuracy and development of the project. In order to find the most suitable one for our case we consulted professors familiar with such techniques as well as books and other resources available on the Internet.

\subsection*{Solution 5}
We started by defining our problem by laying down its tasks, performance measure and training experience, and ended up classifying it as a problem of supervised regression. From there we searched and compared different algorithms that are commonly used to address this type of problem to finally choose Artificial Neural Networks, mainly because it's robust to errors in the training input and has been successfully applied to problems that involve sensor data. Since we had previously decided to work with the Python Programming Language, we ended up using Keras, a highly modular neural networks library, on top of the Theano Python Library.

\end{document} 