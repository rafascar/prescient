\documentclass[12pt]{partialreport}

\projectTitle{Prescient: Context-Aware Home \\Monitoring and Control Central\\}
\idJems{155008}
\professorName{Antônio Augusto Medeiros Fröhlich}
\dateReport{04/06/2016}
\studentsNames{Rodrigo Schmitt Meurer\\Rafael Eriberto Mariot Scarduelli \\ Daniel Manzoni Seerig}
\university{Federal University of Santa Catarina}
\proposedSchedule {%
04/04 & Send proposal \\ 
20/05 & Receive board \\
20/05 & Start documentation \\
20/05 - 10/06 & Development of Galileo's case with board sensors \\
20/05 - 25/05 & Board Sensors Implementation and Tests \\
25/05 - 30/05 & Integration of Galileo with LISHA's Smart Room \\
01/06 - 10/06 & Integration Validation \\
10/06 - 15/07 & Hybrid application Development for smart devices control \\
15/07 - 20/07 & Implementation and Evaluation of user on network discovery \\
20/07 - 30/07 & Interface Galileo Gateway with Web and Database \\
01/08 & Start development of context-aware algorithm \\
21/08 & Start development of data visualization in hybrid application \\
20/09	& Finish data visualization in hybrid application \\
20/09	& Send partial report \\
05/10	& Conclude context-aware algorithm \\
06/10 - 16/10  & Final Tests, bug corrections and implementation validation \\ 
20/10	& Prepare for presentation \\ 
20/10 & Finish documentation  \\ 
30/10	& Submit final report \\
01/11 - 04/11 & Presentation at SBESC \\
}

\updatedSchedule{%
04/04 & Send proposal & 100 \% \\ 
20/05 & Receive board & 100 \%\\
20/05 & Start documentation & 25 \%\\
22/05 - 30/05 & Development of hybrid application base graphical interfaces based on HTML, JavaScript and Jquery Mobile & 100 \%\\
01/06 - 05/06 & Integration and validation of the graphical interface with charting libraries for data visualization & 100 \%\\
05/06 - 07/06 & Study of board capabilities and embedded operating system choice & 100 \%\\
07/06 - 10/06 & Development of smartphone detection daemon on PC& 100 \%\\
08/06 & Technical Webinar & 100 \%\\
10/06 - 20/06 & Ubilinux installation and setup on Galileo & 100 \%\\
20/06 & Setup subnet and wireless router to test the daemon on Galileo & 100 \%\\
21/06 & Primary tests for the smartphone detection in Galileo (Proof of Concept) & 100 \%\\
22/06 - 30/06 & First tests with Galileo and EPOSMoteIII (attempting to control LISHA's Smart Room lights) & 100 \%\\
01/07 - 10/07 & Evaluation of desired sensors to attach to Galileo and EPOSMoteIII & 100 \%\\
10/07 - 12/07 & Instalation and tests with wiringx86 python module for Galileo Gen2 & 100 \%\\
13/07 - 25/07 & Sensor Tests on Galileo Board with wiringx86 & 80 \%\\
30/07 & Start development of context-aware algorithm & 30 \%\\
01/08 - 08/08 & Communicate hybrid application with python in Galileo through Flask API &  80\% \\
08/08 - 12/08 & Performance evaluation for the user detection daemon & 100 \% \\
24/08 & Technical Webinar & 100 \%\\
01/09 - 10/09 & Construction of Galileo and EPOSMoteIII Case & 90 \%\\
11/09 & Galileo and EPOSMoteIII installation on Case & 100 \% \\
12/09 & Design printed circuit to handle the sensors and other peripherals & 20 \%\\
15/09 & Start construction of Smart Room Model for the Demo at SBESC & 15 \%\\
20/09 & Partial report submission & 100 \%\\
20/09 & Reinstall AC control with smart device & -\\
21/09 - 28/09 & Finish hybrid application development & -\\
29/09 - 07/10 & Train and test algorithm with real data & -\\
08/10 - 10/10 & Finish refining machine learning algorithm & - \\
11/10 & Validate results and overall implementation & - \\
11/10 & Record video for presentation & - \\
12/10 - 19/10 & Finish demo model of the smart room & - \\
20/10 & Prepare for presentation & - \\ 
20/10 - 28/10 & Finish documentation & - \\ 
30/10 & Submit final report & - \\
}

\begin{document}
\maketitle
\content

\section*{Proposed Scheduled}
\oldScheduleTable

\section*{Update Scheduled}
\newScheduleTable

\section*{Difficulties and Workarounds}
\subsection*{Situation 1}
In our project we proposed to identify the user in the smart room by the MAC address of it's cellphone. To do that we chose to use the NMap tool that is able to show all the MAC addresses of the devices connected to the network. At first we tried using the system call from the arduino IDE to use linux's functions.
\subsection*{Solution 1}
Instead of using the linux image we chose to use an alternative embedded linux distribution based on debian, called ubilinux. ubilinux is an embedded Linux distribution from Emutex, based on Debian Jessie. It is targeted at embedded devices that have limited memory and storage capabilities. This Linux distribution uses the same package manager as Ubuntu, Debian and other distributions, the apt-get. Therefore this distribution has access to a much wider range of packages available for linux. This way we were able to sucessefully use the Nmap tool, thus enabling us to detect the smartphone in the network.

\subsection*{Situation 2}
After installing Ubilinux in the galileo board we were unable to program the GPIO pins of the board or program anything via the arduino interface. 

\subsection*{Solution 2}
There is a python module that allows python to access the GPIO pins of the Galileo board. This library is developed by Emutex and provides a simple and unified API to talk to the GPIO pins in the Intel Arduino Capable boards. Up to now this module supports: Writing to a GPIO pin configured as output, Reading from a GPIO pin configured as high impedance input, Reading from a GPIO pin configured as pullup input, Reading from a GPIO pin configured as pulldown input, Reading from a GPIO pin configured as analog input (ADC), Writing to a GPIO pin configured as analog output (PWM). Since our project does not use any further capabilities 

\subsection*{Situation 3}
Flask
\subsection*{Solution 3}


\subsection*{Situation 4}
On our first try to detect the smartphone with a certain MAC address connected to the network we implemented a bash script on a PC that uses the Nmap tool to verify if this certain address is connected or not to the network. In the PC we did not encounter any performance related problems. However a PC processor is many times faster than a embedded such as the one used on the Galileo Board. Since user detection is a process that must be running with high frequency to detect if the user is present of left the network, when running it on Galileo it makes the processor busy for much time, reducing the overall performance of the system.

\subsection*{Solution 4}
In order to deal with this performance issue we noted that we coul


\subsection*{Situation 5}
Machine learing

\subsection*{Solution 5}
Keras and tensorFLow

\end{document} 